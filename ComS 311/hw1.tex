\documentclass{article}

\usepackage{amsmath}
\usepackage{amssymb}
\usepackage{graphicx}
\usepackage{epsfig}

% epsf figures (inline or centered)
\newcommand{\FIG}[1]{\begin{center}
  \mbox{\epsfclipoff\epsffile{#1.eps}}
  \end{center}}
\def\epsfsize#1#2{1.0#1}	      % change 1.0 to any zoom factor you want

% numeric sets
\newcommand{\Reals}{\mathbb{R}}      % real numbers
\newcommand{\Naturals}{\mathbb{N}}   % natural numbers
\newcommand{\Integers}{\mathbb{Z}}   % integer numbers
\newcommand{\Rationals}{\mathbb{Q}}  % rational numbers
\newcommand{\Complexes}{\mathbb{C}}  % complex numbers
\newcommand{\ra}{\Rightarrow}
 \begin{document}
 	\title{COM S 311\large \\Homework 1\\Recitation 5, 1-2pm, Marios Tsekitsidis}
	\author{Christian Shinkle}
	\maketitle
	\begin{enumerate}
		\item Base Case: Note
		\[F^2_1=F_1*F_1\]
		\[1^2=1*1\]
		\[1=1\]
		Inductive Hypothesis: $\forall n\leq 1, F^2_1+F^2_2+F^2_3+...+F^2_n=F_n*F_{n+1}$\\
		Goal: Prove $F^2_1+F^2_2+F^2_3+...+F^2_n+F^2_{n+1}=F_{n+1}*F_{n+2}$\\
		Inductive Case: By I.H., 
		$F^2_1+F^2_2+F^2_3+...+F^2_n+F^2_{n+1}=F_n*F_{n+1}+F^2_{n+1}$\\
		$=F_{n+1}(F_n+F_{n+1})$\\
		$=F_{n+1}*F_{n+2}$ by definition of Fibonacci Numbers\\ 
		\item
		\begin{enumerate}
		\item 
		\item Base Case: A $T$ which is a single node is a leaf, so \[i(T)=(n(T)-1)2=(1-1)/2=0\]\\
		Inductive Hypothesis: Let $X$ and $Y$ be two FBT's with $i(X)=(n(x)-1)/2$ and 
		$i(Y)=(n(Y)-1)/2$.\\
		Goal: For tree $R$, $i(R)=(n(R)-1)/2$.\\
		Inductive Case: Let $R$ be a FBT with root $r$ and left child FBT $X$ and right child FBT $Y$. 
		Then, $i(R)=i(X)+i(Y)+1$ because $R$ consists of all nodes of $X$ and $Y$ plus the root $r$.
		By I.H., \\$i(R)=(n(X)-1)/2+(n(Y)-1)/2)+1$. Then, $i(R)$\\
		$=\frac{(n(X)-1+n(Y)-1}{2}+1$\\
		$=\frac{(n(X)+n(Y)-2}{2}+1$\\
		$=\frac{(n(R)-1)-2}{2}+1$ by definition of FBT\\
		$=(n(R)-1)/2-1+1$\\
		$=(n(R)-1)/2$\\
		\end{enumerate}
		\item Base Case: Note $x=a, m=n, y=1$. Therefore, \\
		$a^n=x^{m_0}_0*y_0$\\
		$=a^n*1$\\
		Inductive Hypothesis: $\forall i, a^n=x^{m_i}_i*y_i$
		where is the number of iterations through the loop.\\
		Inductive Case: At the begining of the $i+1th$ iteration through the loop, 
		one of two cases can arise, $m$ is even or odd.\\
		Case 1: $m$ is even.  Then, \\
		$x_{i+1}=x_i^2$, $m_{i+1}=m_i/2$, and $y_{i+1}=y_i$. So,\\
		$x_{i+1}^{m_{i+1}}*y_{i+1}$\\
		$=x_i^{2*m_i/2}*y_i$\\
		$=x_i^{m_i}*y_i$\\
		By I.H., $a^n=x^{m_i}_i*y_i$, therefore the property holds.\\
		Case 2: $m$ is odd. Then, \\
		$y_{i+1}=x_i*y_i$, $x_{i+1}=x_i^2$, and  $m_{i+1}=(m_i-1)/2$. So, \\
		$x_{i+1}^{m_{i+1}}*y_{i+1}$\\
		$=x_i^{2*(m_i-1)/2}*x_i*y_i$\\
		$=x_i^{m_i-1}*x_i*y_i$\\
		$=x_i^{m_i}*y_i$\\
		By I.H., $a^n=x^{m_i}_i*y_i$, therefore the property holds.\\
		Both cases hold, so the property holds for the $i+1th$ iteration.\\
		\item
	\end{enumerate}
 \end{document}