\documentclass{article}

\usepackage{amsmath}
\usepackage{amssymb}
\usepackage{graphicx}
\usepackage{epsfig}
\usepackage[margin=0.9in]{geometry}

% epsf figures (inline or centered)
\newcommand{\FIG}[1]{\begin{center}
  \mbox{\epsfclipoff\epsffile{#1.eps}}
  \end{center}}
\def\epsfsize#1#2{1.0#1}	      % change 1.0 to any zoom factor you want

% numeric sets
\newcommand{\Reals}{\mathbb{R}}      % real numbers
\newcommand{\Naturals}{\mathbb{N}}   % natural numbers
\newcommand{\Integers}{\mathbb{Z}}   % integer numbers
\newcommand{\Rationals}{\mathbb{Q}}  % rational numbers
\newcommand{\Complexes}{\mathbb{C}}  % complex numbers
 \newcommand\tab[1][1cm]{\hspace*{#1}}
 
 \begin{document}
 	\title{COM S 311\large \\Homework 2\\Recitation 5, 1-2pm, Marios Tsekitsidis}
	\author{Christian Shinkle}
	\maketitle
	\begin{enumerate}
		\item %1
		\begin{enumerate}
			\item For the worst-case run time, r =\\
			$=\sum_{i=1}^{n}\sum_{j=i}^{n}\sum_{k=1}^{j-i}c$\\
			$=\sum_{i=1}^{n}\sum_{j=i}^{n}(j-i)c$\\
			$=\sum_{i=1}^{n}c\sum_{j=i}^{n}(j-i)$ by property of summations\\
			$=\sum_{i=1}^{n}c(\sum_{j=i}^{n}j-\sum_{j=i}^{n}i)$ by property of summations\\
			$=\sum_{i=1}^{n}c(\frac{(n+i)(n-i+1)}{2}-(i(n-i)+1))$\\ 
			$=c\sum_{i=1}^{n}\frac{n^2+n-i^2+i}{2}-(in-i^2+1)$\\
			$=c\sum_{i=1}^{n}\frac{n^2+n-i^2+i}{2}-in+i^2-1$\\
			$=c\sum_{i=1}^{n}\frac{n^2+n-i^2+i-2in+2i^2-2}{2}$\\
			$=c/2\sum_{i=1}^{n}n^2+n+i^2+i-2in-2$\\
			$=c/2(\sum_{i=1}^{n}n^2+\sum_{i=1}^{n}n+\sum_{i=1}^{n}i^2+
			\sum_{i=1}^{n}i-\sum_{i=1}^{n}2in-\sum_{i=1}^{n}2)$\\
			$=c/2(n^3+n^2+(2n^3+3n^2+n)/6+(n^2+n)/2-n^3-n^2-2n)$\\
			$=c/2(\frac{1}{3}n^3+n^2-\frac{4}{3}n)$\\
			Therefore, $c/2(\frac{1}{3}n^3+n^2-\frac{4}{3}n) \in O(n^3)$.
			\item For the worst-case run time, \\
			inner loop = $\sum_{j=1}^ic=i*c$\\
			For the outer loop, note after $k$ iterations, $(((i/2)/2)/2.../2)=1$. So, $i/2^k=1$ which
			implies $k=\log _2i$. So, the outer loop $=\log (ic)\in O(\log i)$.\\
			Because $i=n$ at the start of the outer loop, the worst-case run time is $O(\log n)$.\\
		\end{enumerate}
		\item %2
		\begin{enumerate}
			\item For the worst-case run time, it is $\sum_{i=1}^{n-1}\sum_{j=1}^{i}c$ because
			the for-loop will always run $n-1$ times and the while-loop will have to run from $j=i$ until
			$j=1$ because the maximum number of times it will run is if the array is reversed and when 
			the array is reversed, $a[j-1]$ will always be greater than $a[j]$ so the while-loop will only 
			terminate when $j=0$. Also, note that swap and decrementing $j$ both run in constant 
			time.\\
			Therefore, $\sum_{i=1}^{n-1}\sum_{j=1}^{i}c=$\\
			$c\sum_{i=1}^{n-1}(i+1)$\\
			$\approx c\sum_{i=1}^{n-1}i$\\
			$=c((n-1)n)$\\
			$=c(n^2-n)\in O(n^2)$\\
			
			For best-case run time, it is $\sum_{i=1}^{n-1}c$ because the for-loop will always run 
			$n-1$ times and the while-loop will never run because $a[j-1]$ will ways be less than $a[j]$ 
			so the algorithm will never enter it. Note, this means $j=1$ and checking $a[j-1] > a[j]$ 
			will run in constant time.
			Therefore, $\sum_{i=1}^{n-1}c$\\
			$=c(n-1)\in O(n)$\\
			\item The run times of the algorithms are as follows:\\
			\center Sorted Arrays
			\begin{center}
			\begin{tabular}{ |c|c|c|c|}
				\hline
				 & 3000 & 30000 & 300000\\
				Selection & 6 & 84 & 8069\\
				Bubble & 0 & 0 & 2\\
				Insertion & 0 & 1 & 2\\
				\hline
			\end{tabular}
			\end{center}
			
			\center Reverse Sorted Arrays
			\begin{center}
			\begin{tabular}{ |c|c|c|c|}
				\hline
				 & 3000 & 30000 & 300000\\
				Selection & 12 & 1098 & 108541\\
				Bubble & 14 & 432 & 455\\
				Insertion & 10 & 455 & 44862\\
				\hline
			\end{tabular}
			\end{center}

			\center Randomized Arrays
			\begin{center}
			\begin{tabular}{ |c|c|c|c|}
				\hline
				 & 3000 & 30000 & 300000\\
				Selection & 11 & 1115 & 109894\\
				Bubble & 8 & 1167 & 119419\\
				Insertion & 2 & 231 & 23291\\
				\hline
			\end{tabular}
			\end{center}
		\end{enumerate}
		\item %3
		\begin{itemize}
			\item Direct Proof: For  $48n^4-46n^2+25n+31\in O(n^4)$, \\
			$\exists c$ s.t. $\forall n, 48n^4-46n^2+25n+31\leq cn^4$.\\
			Note, $48n^4-46n^2+25n+31$\\
			$\leq 48n^4+46n^4+25n^4+31n^4$\\
			$=150n^4$\\
			Therefore,  $48n^4-46n^2+25n+31\leq cn^4$ for $c=150$, so \\
			$48n^4-46n^2+25n+31\in O(n^4)$\\
			\item Direct Proof:  For $n^{\log n}\in O(2^{\sqrt n})$, 
			$\exists c$ s.t. $\forall n, n^{\log n}\leq c2^{\sqrt n}$.\\
			Using the fact, $\forall k, a>0, \log ^k n\in O(n^a)$, this shows\\ 
			$\log ^2n\in O(\sqrt n)$ which means $\exists c_1$ s.t. $\log ^2n \leq c_1\sqrt n$. \\
			Then, $\log ^2n \leq c_1\sqrt n$\\
			$=2^{\log ^2n}\leq 2^{c_1\sqrt n}$\\
			$=n^{\log n}\leq 2^{c_1\sqrt n}$\\
			Therefore, let $c=2^{c_1}$. This shows $\exists c$ s.t. $\forall n, n^{\log n}\leq 
			c2^{\sqrt n}$. Therefore, $n^{\log n}\in O(2^{\sqrt n})$.\\
			\item Proof by Contradiction: Assume $2^{2^{n+1}}\in O(2^{2^n})$. \\
			Then, $2^{2^{n+1}}\leq c2^{2^n}$. Then,\\
			$2^{2^{n+1}}$\\
			$=2^{2(2^n)}$\\
			$=2^{2^n+2^n}$\\
			$=2^{2^n}*2^{2^n}$\\
			So, $2^{2^n}*2^{2^n}\leq c2^{2^n}$\\
			$2^{2^n}\leq c$\\ This is a contradiction because $c$ cannot outgrow a function of $n$.
			Therefore, $2^{2^{n+1}}\notin O(2^{2^n})$.\\
			\item Proof by Contradiction: Assume $n^3(5+\sqrt n)\in O(n^3)$. Then,\\
			$n^3(5+\sqrt n)\leq cn^3$\\
			$5+\sqrt n\leq c$. This is a contradiction because $c$ cannot outgrow a function of $n$.
			Therefore, $n^3(5+\sqrt n)\notin O(n^3)$.\\
		\end{itemize}
		\item %4
		\begin{enumerate}
			\item The algorithm for calculating change from base $k$ to $10$ is the following:\\
			$n =$ number of digits in base-$k$ number\\
			$result = 0$\\
			for $i$ in the range $[0,n-1]$\\
			$\tab result = result + pow(k, i)*digits[n-i-1]$\\
			return $result$\\
			The algorithm for calculating $pow$ is the following:\\
			Give $a, i$,\\
			$x=a$ $m=i$ $y=1$\\
			while $m>1$\\
			$\tab$ if ($m$ is odd)\\
			$\tab \tab y=x*y$\\
			$\tab \tab m=(m-1)/2$\\
			$\tab$ else if $m$ is even\\
			$\tab \tab m=m/2$\\
			$\tab x=x*x$\\
			return $(x*y)$\\
			For the run time analysis, we start with run time for $pow$:\\
			Let $k$ be the number of iterations through the while-loop.\\
			Then, $4k+3$ is the number of operation. Note $k=\log i$.\\
			Therefore, $pow$ runs in O($\log i$) where $i$ is the exponent of the power.\\
			Using this information, the run time of the full algorithm is:\\
			$\sum_{i=0}^{n-1}$*O($\log i$)\\
			$=n*\log i\in O(n\log n)$\\
			\item The algorithm of calculating base $k$ number from decimal number $m$:\\
			Inputes: $k,m$ where $k$ is the base of the number being calculated and $m$ is the 
			decimal number being converted from.\\
			Let $j$ = number of digits of the output number and let $output$ be an array who's 
			indices represent the digits of the number returned by the algorithm from the most 
			significant bit to the least significant bit.\\
			 for $i$ in range [$0, j-1$]\\
			 $\tab output[j-i-1] = m \mod k$\\
			 $\tab m=m/k$\\
			 return $output$\\
			Note $j=1+\log m$. This means the run time of this algorithm is:\\
			$\sum_{i=0}^{\log m}c=\log (m)c\in O(\log m)$\\
		\end{enumerate}
	\end{enumerate}
 \end{document}