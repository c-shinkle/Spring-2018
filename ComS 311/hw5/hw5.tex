\documentclass{article}

\usepackage{amsmath}
\usepackage{amssymb}
\usepackage{graphicx}
\usepackage{epsfig}

% epsf figures (inline or centered)
\newcommand{\FIG}[1]{\begin{center}
  \mbox{\epsfclipoff\epsffile{#1.eps}}
  \end{center}}
\def\epsfsize#1#2{1.0#1}	      % change 1.0 to any zoom factor you want

% numeric sets
\newcommand{\Reals}{\mathbb{R}}      % real numbers
\newcommand{\Naturals}{\mathbb{N}}   % natural numbers
\newcommand{\Integers}{\mathbb{Z}}   % integer numbers
\newcommand{\Rationals}{\mathbb{Q}}  % rational numbers
\newcommand{\Complexes}{\mathbb{C}}  % complex numbers
\newcommand\tab[1][5mm]{\hspace*{#1}}
 \begin{document}
 	\title{COM S 311\large \\Homework 5\\Recitation 5, 1-2pm, Marios Tsekitsidis}
	\author{Christian Shinkle}
	\maketitle
	\begin{enumerate}
		\item If we place every vertices in $V$ into a priority queue, back by a max-hep, and 
		pritortize them based on the size of their degree. We can reduce the runtime of this algorithm
		$O(nlogn)$ where $n$ is the number of vertices in $V$.\\
		\item 
		\begin{enumerate}
			\item The algorithm is as follows:\\
			Input: Graph $G$ and Set $S$\\
			Let $visited$ be an array of size $|S|$ where all elements are 0.\\
			for each $s\in S$\\
			$\tab$ for each $e$ that is adjacent to $s$\\
			$\tab \tab$ if $visited[e] ==0$\\
			$\tab \tab \tab$ $visited[e]\gets 1$\\
			$\tab \tab$ else \\
			$\tab \tab \tab$ return false\\
			return true\\
			The largest $S$ can be is $|V|=n$, so it will take $O(n)$. The outer loop will take $n$ 
			time and the inner loop will take $n$ time, so the loops will take $O(n^2)$ time. 
			Therefore the runtime is $O(n)+O(n^2)\in O(n^2)$.\\
			\item The algorithm is as follows:\\
			Input: Graph $G$\\
			Calculate all $2^n$ combinations of independent sets\\
			Let $S$ be the largest of all the sets\\
			return $S$\\
			This problem is NP, so the runtime will be at least in $2^N$. It will take $n$ time to 
			calculate a single independent set and there are $2^n$ number of sets possible in the 
			worst case, so this algorithm is $\in O(2^n)$.\\ 
		\end{enumerate}
		\item The algorithm is as follows:\\
		Input: $n\times m$ matrix $M$\\
		Let $C$ be an $n\times m$ matrix that has all cells initialized to $-\infty$.\\
		for $i$ in the range 1 to $n$\\
		$\tab$ $C[i,1]\gets M[i,1] $\\
		for $i$ in the range 1 to $n$\\
		$\tab$ for $j$ in the range 2 to $m$\\
		$\tab \tab$ if $i-1$ is in bounds and $M[i-1,j-1]+M[i,j]>C[i,j]$\\
		$\tab \tab \tab C[i,j]\gets M[i-1,j-1]+M[i,j]$\\
		$\tab \tab$ if $M[i,j-1]+M[i,j]>C[i,j]$\\
		$\tab \tab \tab C[i,j]\gets M[i-1,j-1]+M[i,j]$\\
		$\tab \tab$ if $i+1$ is in bounds and $M[i+1,j-1]+M[i,j]>C[i,j]$\\
		$\tab \tab \tab C[i,j]\gets M[i+1,j-1]+M[i,j]$\\
		return the max value in the $m$ column of $C$.\\
		The recurrence relationship is as follows:\\
		$T(i,j)=$ max$(T(i-1,j-1),\ T(i,j-1),\ T(i+1,j-1))+M[i,j]$\\
		The runtime of the iterative algorithm is as follows:\\
		The big-O of initailizing $C$ is $O(nm)$ because there are $n\times m$ cells in $C$ and 
		every cell will be visisted. The big-O of the first loop is $n$ trivially. The big-O of the nested
		for loops is $O(nm)$ because every cell will be visited 4 times and the visiting of the cell
		will take a constant number of opertation. Lastly, the final loop will take $O(n)$ trivially.
		Therfore, the runtime of the algorithm is $cnm+cn+cnm+cn\in O(nm)$.\\
		\item  The algorithm is as follows:\\
		Input : set $S$ of $n$ non-negative integers\\
		Let $N$ be the sum of all $x_i\in S$.\\
		if $N$ is odd, then return false and terminate\\
		Let $N'$ be 0.\\
		Let $best\gets N$\\
		While $best\neq 0$\\
		$\tab$ Let $min$ be $\infty$.\\
		$\tab$ Let $index$ be 1.\\
		$\tab$ for $i$ in the range 1 to $n$\\
		$\tab \tab$ if $|best-2*S[i]|<min$\\
		$\tab \tab \tab$ $min\gets |best-2*S[i]|$\\
		$\tab \tab \tab$ $index\gets i$\\
		$\tab$ if $min<best$\\
		$\tab \tab$ $N\gets N-S[index]$\\
		$\tab \tab$ $N'\gets N'+S[index]$\\
		$\tab \tab$ $best\gets min$\\
		$\tab \tab$ remove $S[index]$ from $S$\\
		$\tab$ else\\
		$\tab \tab$ return false and terminate\\
		return true\\ \\
		The runtime of the iterative algorithm is as follows:\\
		Calculating the sume of all $x_i\in S$ will take $N$ time. The while loop will run at most
		$n$ times. The for loop will run at most $n$ times. Therefore, the runtime is $O(N+n^2)$.\\
		The recurrence relationship is as follows:\\
		$T(n) = T(n-1) + n$.
	\end{enumerate}
 \end{document}