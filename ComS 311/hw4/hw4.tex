\documentclass{article}

\usepackage{amsmath}
\usepackage{amssymb}
\usepackage{graphicx}
\usepackage{epsfig}

% epsf figures (inline or centered)
\newcommand{\FIG}[1]{\begin{center}
  \mbox{\epsfclipoff\epsffile{#1.eps}}
  \end{center}}
\def\epsfsize#1#2{1.0#1}	      % change 1.0 to any zoom factor you want

% numeric sets
\newcommand{\Reals}{\mathbb{R}}      % real numbers
\newcommand{\Naturals}{\mathbb{N}}   % natural numbers
\newcommand{\Integers}{\mathbb{Z}}   % integer numbers
\newcommand{\Rationals}{\mathbb{Q}}  % rational numbers
\newcommand{\Complexes}{\mathbb{C}}  % complex numbers
\newcommand\tab[1][5mm]{\hspace*{#1}}
 \begin{document}
 	\title{COM S 311\large \\Homework 4\\Recitation 5, 1-2pm, Marios Tsekitsidis}
	\author{Christian Shinkle}
	\maketitle
	\begin{enumerate}
		\item 
		\begin{enumerate}
			\item $T(1)=1$; $T(n)$\\
			$=3T(n/2)+n$\\
			$=3(3T(n/2^2)+n/2^1)+n=3^2T(n/4)+3/2n+n$\\
			$=3(3^2T(n/2^3)+3/2n/2+n/2)+n=3^3T(n/8)+(3/2)^2n+3/2n+n$\\
			$=3^kT(n/2^k)+n\sum^{k-1}_{i=0}(3/2)^i$ where $k=\log_2 n$\\
			$=3^{\log n}T(n/n)+n\sum^{\log n-1}_{i=0}(3/2)^i$\\
			$=n^{\log 3}+n\sum^{\log n-1}_{i=0}(3/2)^i$\\
			$=n^{\log 3}+n(\frac {1-(3/2)^{\log n}}{1-3/2})$\\
			$=n^{\log 3}-2n(1-(3/2)^{\log n})$\\
			$=n^{\log 3}-2n(1-n^{\log 3/2})$\\
			$=n^{\log 3}-2n(1-n^{\log (3) -1 })$\\
			$=n^{\log 3}-2n+2n*n^{\log (3) -1 }$\\
			$=n^{\log 3}-2n+2n^{\log 3}$\\
			$=3n^{\log 3}-2n$\\
			
			\item $T(1)=1$; $T(n)$\\
			$=T(n/8)+n$\\
			$=T(n/8^2)+n+n/8$\\
			$=T(n/8^3)+n+n/8+n/8^2$\\
			$=T(n/8^k)+n\sum_{i=0}^{k}(1/8)^i$, where $k=\log_8 n$\\
			$=1+n(\frac{1-(1/8)^{\log_8 n+1}} {7/8})$\\
			$=1+\frac 8 7 n(1-\frac 1{8n})$\\
			$=\frac 8 7 n+\frac 6 7$\\
			
		\end{enumerate}
		\item BuildBST:\\
		Input: Sorted Array $arr$\\
		Integer $length$ = length of $arr$\\
		if arr is empty\\
		$\tab$ return null\\
		Node $root.data$ = $arr[length/2]$ \\
		$root.left$ = BuildBST(subarray of $arr$ from indices 0 to $length/2-1$)\\
		$root.right$ = BuildBST(subarray of $arr$ from indices \\
		$\tab \tab \tab \tab length/2+1$, $length-1$)\\
		return root\\ \\
		Note that when the indices for subarray are out of bounds(i.e. the starting index is greater than 			the end index), it will create an empty arr.\\ \\
		Recurrence relation: $T(n) = 2T(\frac{n-1}{2})+c_1$ , $T(1)=1$.\\
		Note $T(n) = 2T(\frac{n-1}{2})+c_1\leq 2T(\frac{n}{2})+c_1$, so we will use the second function to analize the runtime.\\				
		Then, $T(n) = 2T(\frac{n}{2})+c_1$\\
		$= 2^1(2T(\frac{n}{2^2})+c_1)+c_1)=2^2T(\frac{n}{2^3})+3c_1$\\
		$= 2^2(2T(\frac{n}{2^3})+c_1)+3c_1)=2^3T(\frac{n}{2^3})+7c_1$\\
		$= 2^3(2T(\frac{n}{2^4})+c_1)+7c_1)=2^4T(\frac{n}{2^4})+15c_1$\\
		$=2^kT(\frac{n}{2^k})+(2^k-1)c_1$, where $k=$ number of iterations.\\
		Note $1=n/2^k$ implies $k=\log n$, so \\
		$2^{\log n}T(n/2^{\log n})+(2^{\log n}-1)c_1$\\
		$=n+(n - 1)c_1\in O(n)$\\
		\item I don't know how to solve this problem.\\
		\item For $G'$, $E'$ will be an adjacency list composed of an Array of Lists. The size of the
		Array will be the number of vertices in $G$. In order to add to the adjacency list, the method
		add() will be used, which takes two arguments:Vertex $u$, Vertex $v$, where $u$ is the 
		location in the Array in the adjacency list and $v$ is the Vertex that is added to $E'$.
		\\ \\
		Algorithm:\\
		Input: Adjacency List $G$\\
		Create HashTable $h$ which holds pairs of Vertices\\
		Create $G'$ with $V'$ being a copy of $V$ and an empty adjacency list for $E'$\\
		for each Vertex $a$ in the Array of $G$\\
		$\tab$for each Vertex $b$ in the List at $a$ in the Array of $G$\\
		$\tab \tab$ for each Vertex $c$ in the List at $b$ in the Array of $G$\\
		$\tab \tab$ if $h$ doesn't contain pair $<a,b>$\\
		$\tab \tab \tab $ add $<a,b>$ to $h$\\
		$\tab \tab \tab $ $G'$.add($u$, $v$)\\
		\\
		return $G'$\\
		\\
		Runtime:\\
		Preparing the HashTable will take constant time and copying $V$ into $V'$ will take $n$ time.
		The outer for-loop runs $n$ times because it check every Vertex in the Array of G. The 
		middle-loop and inner-loop runs $m$ times each because, in the worst case, the Array at $a$
		may contain $m$ edges. The if-statement takes constant time because, in order to check if a
		HashTable contains an element, it requires a search of the HashTable, which takes 
		average/expect constant time. Adding to a HashTable takes average/expect constant time and 
		the add function for $G'$ takes constant time because accessing an Array takes constant time
		and adding to a List also takes constant time. So, the inner most code takes a constant amount
		of time. The following expression describes the runtime:\\
		$n+\sum_{i=0}^n\sum_{j=0}^m\sum_{k=0}^mc$\\
		$=n+nm^2c\in O(nm^2)$\\
	\end{enumerate}
 \end{document}